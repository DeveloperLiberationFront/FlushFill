\documentclass[11pt]{article}
\usepackage{cite}
\usepackage{marginnote}

\renewcommand*{\marginfont}{\footnotesize}

%opening
\title{Approaches to Transparent Program Synthesis in Excel}
\author{Justin A. Middleton\\North Carolina State University, Raleigh, USA\\jamiddl2@ncsu.edu}

\begin{document}

\maketitle

\begin{abstract}
Program synthesis in Excel grants spreadsheet users a powerful way of transforming
input without much work. However, users have no insight into what programs are being
made and, therefore, they can't apply automatically generated solutions to situations
or learn programs on their own. My work here is to meet the program synthesizing
capabilities of current spreadsheets but make the resulting program accessible to
the user, both in visibility and in comprehensibility. In the process, I evaluate
a number of approaches to the this problem of transparent synthesis, ranging from
the algorithms of FlashFill itself to principles of planning.
\end{abstract}

\section{Introduction}
\marginnote{I took these words straight from my 582 pitch -- no fears, will change.}
Millions of programmers rely on spreadsheet programs, like Microsoft Excel, to make sense of their data, and, as part of some of my other ongoing research outside of class, I’m looking for ways to make their lives easier. Already shipped with Excel is a feature called Flash Fill – given a set of inputs and a few sample outputs, it tries to create a program which captures the pattern of transformation which maps the set of inputs to outputs and then applies it to the rest of the incomplete values. However, it does this opaquely; users don’t have insight into the program created, which impairs deep comprehension and reusability of the program.
To be clearer, an example from the paper of FlashFill~\cite{gulwani2011automating} asks about creating a program which captures the following two columns:

\begin{center}
	\begin{tabular}{|c|c|}
		\hline
		Input & Output \\
		\hline\hline
		John DOE 3 Data [TS]865-000-0000 - - 453442-00 06-23-2009 & 865-000-0000 \\
		\hline
		A FF MARILYN 30’S 865-000-0030 4535871-00 07-07-2009 & 865-000-0030 \\
		\hline 
		A GEDA-MARY 100MG 865-001-0020 - - 5941-00 06-23-2009 & 865-001-0020 \\
		\hline
	\end{tabular}
\end{center}

What needs to happen is for a program to discover the correct patterns and manipulations to take the string in the input column and change it to that in the output. In this case, there are many acceptable ways to do this: finding three sets of hyphen-separated numbers with digits in quantities of 3-3-4; finding some hyphen-connected numbers beginning with 865; etc.


\bibliography{FlushFillPaper}{}
\bibliographystyle{plain}

\end{document}
