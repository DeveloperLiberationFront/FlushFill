\documentclass[11pt]{article} \usepackage{cite} \usepackage{marginnote}

\renewcommand*{\marginfont}{\footnotesize}

%opening
\title{FlushFill\\\small{Transparent Program Synthesis in Excel}} \author{Justin A.
	Middleton\\North Carolina State University, Raleigh, USA\\jamiddl2@ncsu.edu}

\begin{document}
	
	\maketitle
	
	\begin{abstract} Program synthesis in Excel grants spreadsheet users a powerful
		way of transforming input without much work. However, users have no insight
		into what programs are being made and, therefore, they can't apply
		automatically generated solutions to situations or learn programs on their own.
		My work here is to meet the program synthesizing capabilities of current
		spreadsheets but make the resulting program accessible to the user, both in
		visibility and in comprehensibility. In the process, I evaluate a number of
		approaches to the this problem of transparent synthesis, ranging from the
		algorithms of FlashFill itself to principles of planning. \end{abstract}
	
	\section{Introduction} \marginnote{I took these words straight from my 582
		pitch -- no fears, will change.} Millions of programmers rely on spreadsheet
	programs, like Microsoft Excel, to make sense of their data, and, as part of
	some of my other ongoing research outside of class, I’m looking for ways to
	make their lives easier. Already shipped with Excel is a feature called Flash
	Fill – given a set of inputs and a few sample outputs, it tries to create a
	program which captures the pattern of transformation which maps the set of
	inputs to outputs and then applies it to the rest of the incomplete values.
	However, it does this opaquely; users don’t have insight into the program
	created, which impairs deep comprehension and reusability of the program. To be
	clearer, an example from the paper of FlashFill~\cite{gulwani2011automating}
	asks about creating a program which captures the following two columns:
	
	\begin{center} \begin{tabular}{|c|c|} \hline Input & Output \\ \hline\hline
			John DOE 3 Data [TS]865-000-0000 - - 453442-00 06-23-2009 & 865-000-0000 \\
			\hline A FF MARILYN 30’S 865-000-0030 4535871-00 07-07-2009 & 865-000-0030 \\
			\hline A GEDA-MARY 100MG 865-001-0020 - - 5941-00 06-23-2009 & 865-001-0020 \\
			\hline \end{tabular} \end{center}
	
	What needs to happen is for a program to discover the correct patterns and
	manipulations to take the string in the input column and change it to that in
	the output. In this case, there are many acceptable ways to do this: finding
	three sets of hyphen-separated numbers with digits in quantities of 3-3-4;
	finding some hyphen-connected numbers beginning with 865; etc.
	
	\section{Approach} \subsection{Phase 1} The fact that the first prototype of
	FlashFill is about 5000 lines of code~\cite{gulwani2011automating} was a good
	cue to me that a full repeat of the code would go out of my time. Even so, the
	algorithm as in that paper offered a great starting point for the
	program-synthesizing novice that I am. To start, then, I chose a plan with
	these few guidelines:
	
	\begin{enumerate}
		
		\item \textbf{A console application in C\#.} Visual Studio has templates for
		Excel add-ins, which would be great to use if not for the extra time it would
		take to learn. Instead, I chose to pipe into the console some CSV files
		mimicking spreadsheet columns. If the algorithm worked out, then, hopefully,
		a move to an Excel add-in (which accepts C\#) would not be too far out.
		
		\item \textbf{None of the Usability extensions.} Let's be clear: I aim to
		prove merely that it's possible, not that I can snap my fingers and make a
		robust interface for semester project.
		
		\item \textbf{Use the examples from paper for tests.} They're already
		written! It's a good way to save me some time.
		
		\item \textbf{Don't worry about planning right now.} Proving I can do it one
		way should be enough for the first phase. Once I can do that, then I can move
		on.
		
	\end{enumerate}
	
	\subsubsection{Major Wrinkles}
	
	\begin{itemize}
		\item[P1] \textbf{A Lack of Loops} The FlashFill algorithm as described in~\cite{gulwani2011automating} defines a use for looping over strings.
		Excel has no function to do the same task.
		
		\item[S1.1] \textit{Fudge it with natural language} Rather than trying
		to do looping a dynamic number of times, just explain what a function
		needs to do with words like "Loop for as many times it has a substring"
		or something.
		
		\item[S1.2] \textit{Go VBA instead.} Try to accomplish the same task
		using VBA lingo over pure Excel functions.
		
		\item[P2] \textbf{A Rejection of Regular Expressions} Don't think I can use these in standard Excel functions either.
		
		\item[S2.1] \textit{Go VBA instead.} Since I think can. Plus, you can still
		use regular Excel functions in VBA\footnote{https://msdn.microsoft.com/en-us/library/office/hh211481\%28v=office.14\%29.aspx}.
	\end{itemize}
	
	\bibliography{FlushFillPaper}{} \bibliographystyle{plain}
	
\end{document}
