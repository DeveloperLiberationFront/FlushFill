\documentclass{sig-alternate-05-2015}
\usepackage{xspace}
\usepackage{hyperref}

\begin{document}

% Copyright
\setcopyright{acmcopyright}
%\setcopyright{acmlicensed}
%\setcopyright{rightsretained}
%\setcopyright{usgov}
%\setcopyright{usgovmixed}
%\setcopyright{cagov}
%\setcopyright{cagovmixed}


% DOI
\doi{???}

% ISBN
\isbn{???}

%Conference
\conferenceinfo{???}{???}

\acmPrice{\$15.00}

%
% --- Author Metadata here ---
\conferenceinfo{???}{???}
%\CopyrightYear{2007} % Allows default copyright year (20XX) to be over-ridden - IF NEED BE.
%\crdata{0-12345-67-8/90/01}  % Allows default copyright data (0-89791-88-6/97/05) to be over-ridden - IF NEED BE.
% --- End of Author Metadata ---

\newcommand{\toolname}{FlushFill\xspace}
\title{Excel Program Synthesis}
%
% You need the command \numberofauthors to handle the 'placement
% and alignment' of the authors beneath the title.
%
% For aesthetic reasons, we recommend 'three authors at a time'
% i.e. three 'name/affiliation blocks' be placed beneath the title.
%
% NOTE: You are NOT restricted in how many 'rows' of
% "name/affiliations" may appear. We just ask that you restrict
% the number of 'columns' to three.
%
% Because of the available 'opening page real-estate'
% we ask you to refrain from putting more than six authors
% (two rows with three columns) beneath the article title.
% More than six makes the first-page appear very cluttered indeed.
%
% Use the \alignauthor commands to handle the names
% and affiliations for an 'aesthetic maximum' of six authors.
% Add names, affiliations, addresses for
% the seventh etc. author(s) as the argument for the
% \additionalauthors command.
% These 'additional authors' will be output/set for you
% without further effort on your part as the last section in
% the body of your article BEFORE References or any Appendices.

\numberofauthors{2} %  in this sample file, there are a *total*
% of EIGHT authors. SIX appear on the 'first-page' (for formatting
% reasons) and the remaining two appear in the \additionalauthors section.
%
\author{
% You can go ahead and credit any number of authors here,
% e.g. one 'row of three' or two rows (consisting of one row of three
% and a second row of one, two or three).
%
% The command \alignauthor (no curly braces needed) should
% precede each author name, affiliation/snail-mail address and
% e-mail address. Additionally, tag each line of
% affiliation/address with \affaddr, and tag the
% e-mail address with \email.
%
% 1st. author
\alignauthor
Justin Middleton\\
       \affaddr{North Carolina State University}\\
       \affaddr{Raleigh, North Carolina, USA}\\
       \email{jamiddl2@ncsu.edu}
% 2nd. author
\alignauthor
Emerson Murphy-Hill\\
       \affaddr{North Carolina State University}\\
       \affaddr{Raleigh, North Carolina, USA}\\
       \email{ermurph3@ncsu.edu}
}

\maketitle
\begin{abstract}
For many spreadsheet users, the full power of Excel goes unrealized because they do not fully grasp how to use spreadsheet functions. In other words, users may know what they want but they don't know how to effect it through programming. To address this, we wrote a example-driven program synthesizer to generate Excel formulae to connect a user's information with intention. We evaluated this program on a curated spreadsheet benchmark to demonstrate the best configuration of heuristics to power this problem.
\end{abstract}


%
% The code below should be generated by the tool at
% http://dl.acm.org/ccs.cfm
% Please copy and paste the code instead of the example below. 
%

%\begin{CCSXML}
%<ccs2012>
% <concept>
%  <concept_id>10010520.10010553.10010562</concept_id>
%  <concept_desc>Computer systems organization~Embedded systems</concept_desc>
%  <concept_significance>500</concept_significance>
% </concept>
% <concept>
%  <concept_id>10010520.10010575.10010755</concept_id>
%  <concept_desc>Computer systems organization~Redundancy</concept_desc>
%  <concept_significance>300</concept_significance>
% </concept>
% <concept>
%  <concept_id>10010520.10010553.10010554</concept_id>
%  <concept_desc>Computer systems organization~Robotics</concept_desc>
%  <concept_significance>100</concept_significance>
% </concept>
% <concept>
%  <concept_id>10003033.10003083.10003095</concept_id>
%  <concept_desc>Networks~Network reliability</concept_desc>
%  <concept_significance>100</concept_significance>
% </concept>
%</ccs2012>  
%\end{CCSXML}
%
%\ccsdesc[500]{Computer systems organization~Embedded systems}
%\ccsdesc[300]{Computer systems organization~Redundancy}
%\ccsdesc{Computer systems organization~Robotics}
%\ccsdesc[100]{Networks~Network reliability}


%
% End generated code
%

%
%  Use this command to print the description
%
\printccsdesc

% We no longer use \terms command
%\terms{Theory}

%\keywords{ACM proceedings; \LaTeX; text tagging}

\section{Introduction}
People use spreadsheets, but they don't really know how to use them. If we insist on using them, then we should at least learn how to use them well.

Much of the functionality of spreadsheets lurks below the interface. Excel's palette of functions is available to anyone brave enough to sift through the documentation but daunting to approach. For many people, it may seem simpler to work with only the bare minimum -- most often SUM and IF -- and often that is all that is necessary. Maybe the reason most people don't use more functions is that they simply are not confident enough to use other functions, not that they are not useful at all.

Some solutions skirt the need for knowing how to program at all. Excel's built-in FlashFill tool already takes much of the guessing out of the work. With it, spreadsheet users can complete their menial work over a column of values by generating a program from a few input-output examples. The user needs no knowledge of the work below the interface. They need only to know what they have and what they want, and this is great for those whom have no time for manual programming.

Nevertheless, it is a learning opportunity lost. FlashFill keeps its solutions to itself, and the user learns only to depend on the black box. What they are not taught is how they might have solved the problem on their own with the tools in front of them -- namely, the toolbox of Excel functions. Instead of ending with only the right answer, the users could be taught the path, if to learn something about how to get there.

%I'm not convinced learning it on their own is the best. Wouldn't we want an "infallible" computer to handle things
%more than a novice. When would a novice sufficiently outperform the computer? Do we want people programming?

%Reapply

We built a tool, \toolname, that maps out this path from problem to solution. Like previous work it generates a program that follows examples to find a new solution for further problems. However, ours also displays this solution in terms of a spreadsheet formula. Once an answer has been found, the user sees how the transformation is accomplished through the combination of spreadsheet functions.

To ensure its usefulness, we then tested \toolname on a benchmark of spreadsheets. First, we wanted to make sure that the formulae created fulfilled the problem to a sufficient degree. Furthermore, we wanted to make sure that the formula, even if correct, were also comprehensible. That is, we want to both help people find the right answer while also finding an answer they can understand and apply.

%Toolname idea: something along maps, pathfinding, etc.

\section{Related Work}
Many approaches to this problem have been proposed~\cite{mitchell1982generalization}\cite{kitzelmann2009inductive}.

%Programming by Examples
Programming by examples has been a goal of the field for years, sparking many approaches in languages such as LISP~\cite{smith1984synthesis}.

Such example-driven synthesis has been used for such things as program code changes based on the edits that programmers make~\cite{DBLP:journals/corr/SousaSDPGGSH16}.

Inductive functional programming~\cite{kitzelmann2008analytical} parallels this research most clearly in the form of MagicHaskeller~\cite{katayama2009recent} and it's more pertinent sibling, MagicExceller\footnote{\href{http://nautilus.cs.miyazaki-u.ac.jp/~skata/MagicExceller.html}{http://nautilus.cs.miyazaki-u.ac.jp/~skata/MagicExceller.html}}.

Elsewhere in spreadsheets, some tools, such as FIDEX, synthesize expressions to filter data in the columns~\cite{Wang:2016:FFS:2983990.2984030}. And FlashFill, already incorporated into Excel, can induce string transformation programs based in input/output examples~\cite{gulwani2011automating}. Unlike \toolname, however, these programs don't display the solution in a domain language which the user can read and adapt elsewhere in the spreadsheet.
%* Spreadsheet

%Programming by Demonstration

Lau and co.'s extension of Mitchell's version space algebra~\cite{mitchell1982generalization} allows them to tackle the problem of repeated text editor tasks~\cite{lau2003programming}.

Programming by natural language has also been successful. Gulwani and Marron's NLyze~\cite{gulwani2014nlyze} combines approaches from keyword programming and semantic parsing to generate the most likely programmings for a given task.

This project also considers the way to present the information, a pertinent question explored by others. Mayer and co. recently explored some new approaches to this problem, suggesting what they call Program Navigation and Conversational Clarification~\cite{Mayer:2015:UIM:2807442.2807459}.

\section{Conclusion}
\toolname fits into an a landscape of spreadsheet solutions while emphasizing a particular approach to the synthesis problem. By scaffolding a framework for heuristics, we have in \toolname a flexible tool that can grow and develop as new ideas come along. That is, the performance that we show in this paper is only one configuration of tested heuristics -- when better ideas come along, we can plug them in and improve performance even further.



%ACKNOWLEDGMENTS are optional
\section{Acknowledgments}

%
% The following two commands are all you need in the
% initial runs of your .tex file to
% produce the bibliography for the citations in your paper.
\bibliographystyle{abbrv}
\bibliography{ExcelSynthesis}  % sigproc.bib is the name of the Bibliography in this case
% You must have a proper ".bib" file
%  and remember to run:
% latex bibtex latex latex
% to resolve all references
%
% ACM needs 'a single self-contained file'!
%
%APPENDICES are optional
%\balancecolumns

\end{document}
